% настройки polyglossia
\setdefaultlanguage{russian}
\setotherlanguage{english}

% локализация
\addto\captionsrussian{
  \renewcommand{\figurename}{Рисунок}%
  \renewcommand{\partname}{Глава}
  \renewcommand{\contentsname}{\centerline{Содержание}}
  \renewcommand{\listingscaption}{Листинг}
}

% основной шрифт документа
\setmainfont{CMU Serif}
\newfontfamily\cyrillicfont{CMU Serif}[Script=Cyrillic]

% перечень использованных источников
\addbibresource{refs.bib}

% настройка полей
\geometry{top=2cm}
\geometry{bottom=2cm}
\geometry{left=2cm}
\geometry{right=2cm}
\geometry{bindingoffset=0cm}

% настройка ссылок и метаданных документа
\hypersetup{unicode=true,colorlinks=true,linkcolor=red,citecolor=green,filecolor=magenta,urlcolor=cyan,        		       
    pdftitle={\docname},   	    
    pdfauthor={\studentname},      
    pdfsubject={\subject},      		        
    pdfcreator={\studentname}, 	       
    pdfproducer={Overleaf}, 		     
    pdfkeywords={\subject}
}

% настройка подсветки кода и окружения для листингов
\usemintedstyle{colorful}
\newenvironment{code}{\captionsetup{type=listing}}{}

% шрифт для листингов с лигатурами
\setmonofont{FiraCode-Regular.otf}[
    SizeFeatures={Size=10},
    Path = templates/,
    Contextuals=Alternate
]

% оформления подписи рисунка
\captionsetup[figure]{labelsep = period}

% подпись таблицы
\DeclareCaptionFormat{hfillstart}{\hfill#1#2#3\par}
\captionsetup[table]{format=hfillstart,labelsep=newline,justification=centering,skip=-10pt,textfont=bf}

% путь к каталогу с рисунками
\graphicspath{{fig/}}

% Внесение titlepage в учёт счётчика страниц
\makeatletter
\renewenvironment{titlepage} {
 \thispagestyle{empty}
}
\makeatother

\counterwithin{figure}{section}
\counterwithin{table}{section}

\titlelabel{\thetitle.\quad}

% для удобного конспектирования математики
\mathtoolsset{showonlyrefs=true}
\theoremstyle{plain}
\newtheorem{theorem}{Теорема}[section]
\newtheorem{proposition}[theorem]{Утверждение}
\theoremstyle{definition}
\newtheorem{corollary}{Следствие}[theorem]
\newtheorem{problem}{Задача}[section]
\theoremstyle{remark}
\newtheorem*{nonum}{Решение}

% настоящее матожидание
\newcommand{\MExpect}{\mathsf{M}}

% объявили оператор!
\DeclareMathOperator{\sgn}{\mathop{sgn}}

% перенос знаков в формулах (по Львовскому)
\newcommand*{\hm}[1]{#1\nobreak\discretionary{} {\hbox{$\mathsurround=0pt #1$}}{}} 
